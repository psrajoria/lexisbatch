% Options for packages loaded elsewhere
\PassOptionsToPackage{unicode}{hyperref}
\PassOptionsToPackage{hyphens}{url}
\documentclass[
]{article}
\usepackage{xcolor}
\usepackage[margin=1in]{geometry}
\usepackage{amsmath,amssymb}
\setcounter{secnumdepth}{-\maxdimen} % remove section numbering
\usepackage{iftex}
\ifPDFTeX
  \usepackage[T1]{fontenc}
  \usepackage[utf8]{inputenc}
  \usepackage{textcomp} % provide euro and other symbols
\else % if luatex or xetex
  \usepackage{unicode-math} % this also loads fontspec
  \defaultfontfeatures{Scale=MatchLowercase}
  \defaultfontfeatures[\rmfamily]{Ligatures=TeX,Scale=1}
\fi
\usepackage{lmodern}
\ifPDFTeX\else
  % xetex/luatex font selection
\fi
% Use upquote if available, for straight quotes in verbatim environments
\IfFileExists{upquote.sty}{\usepackage{upquote}}{}
\IfFileExists{microtype.sty}{% use microtype if available
  \usepackage[]{microtype}
  \UseMicrotypeSet[protrusion]{basicmath} % disable protrusion for tt fonts
}{}
\makeatletter
\@ifundefined{KOMAClassName}{% if non-KOMA class
  \IfFileExists{parskip.sty}{%
    \usepackage{parskip}
  }{% else
    \setlength{\parindent}{0pt}
    \setlength{\parskip}{6pt plus 2pt minus 1pt}}
}{% if KOMA class
  \KOMAoptions{parskip=half}}
\makeatother
\usepackage{color}
\usepackage{fancyvrb}
\newcommand{\VerbBar}{|}
\newcommand{\VERB}{\Verb[commandchars=\\\{\}]}
\DefineVerbatimEnvironment{Highlighting}{Verbatim}{commandchars=\\\{\}}
% Add ',fontsize=\small' for more characters per line
\usepackage{framed}
\definecolor{shadecolor}{RGB}{248,248,248}
\newenvironment{Shaded}{\begin{snugshade}}{\end{snugshade}}
\newcommand{\AlertTok}[1]{\textcolor[rgb]{0.94,0.16,0.16}{#1}}
\newcommand{\AnnotationTok}[1]{\textcolor[rgb]{0.56,0.35,0.01}{\textbf{\textit{#1}}}}
\newcommand{\AttributeTok}[1]{\textcolor[rgb]{0.13,0.29,0.53}{#1}}
\newcommand{\BaseNTok}[1]{\textcolor[rgb]{0.00,0.00,0.81}{#1}}
\newcommand{\BuiltInTok}[1]{#1}
\newcommand{\CharTok}[1]{\textcolor[rgb]{0.31,0.60,0.02}{#1}}
\newcommand{\CommentTok}[1]{\textcolor[rgb]{0.56,0.35,0.01}{\textit{#1}}}
\newcommand{\CommentVarTok}[1]{\textcolor[rgb]{0.56,0.35,0.01}{\textbf{\textit{#1}}}}
\newcommand{\ConstantTok}[1]{\textcolor[rgb]{0.56,0.35,0.01}{#1}}
\newcommand{\ControlFlowTok}[1]{\textcolor[rgb]{0.13,0.29,0.53}{\textbf{#1}}}
\newcommand{\DataTypeTok}[1]{\textcolor[rgb]{0.13,0.29,0.53}{#1}}
\newcommand{\DecValTok}[1]{\textcolor[rgb]{0.00,0.00,0.81}{#1}}
\newcommand{\DocumentationTok}[1]{\textcolor[rgb]{0.56,0.35,0.01}{\textbf{\textit{#1}}}}
\newcommand{\ErrorTok}[1]{\textcolor[rgb]{0.64,0.00,0.00}{\textbf{#1}}}
\newcommand{\ExtensionTok}[1]{#1}
\newcommand{\FloatTok}[1]{\textcolor[rgb]{0.00,0.00,0.81}{#1}}
\newcommand{\FunctionTok}[1]{\textcolor[rgb]{0.13,0.29,0.53}{\textbf{#1}}}
\newcommand{\ImportTok}[1]{#1}
\newcommand{\InformationTok}[1]{\textcolor[rgb]{0.56,0.35,0.01}{\textbf{\textit{#1}}}}
\newcommand{\KeywordTok}[1]{\textcolor[rgb]{0.13,0.29,0.53}{\textbf{#1}}}
\newcommand{\NormalTok}[1]{#1}
\newcommand{\OperatorTok}[1]{\textcolor[rgb]{0.81,0.36,0.00}{\textbf{#1}}}
\newcommand{\OtherTok}[1]{\textcolor[rgb]{0.56,0.35,0.01}{#1}}
\newcommand{\PreprocessorTok}[1]{\textcolor[rgb]{0.56,0.35,0.01}{\textit{#1}}}
\newcommand{\RegionMarkerTok}[1]{#1}
\newcommand{\SpecialCharTok}[1]{\textcolor[rgb]{0.81,0.36,0.00}{\textbf{#1}}}
\newcommand{\SpecialStringTok}[1]{\textcolor[rgb]{0.31,0.60,0.02}{#1}}
\newcommand{\StringTok}[1]{\textcolor[rgb]{0.31,0.60,0.02}{#1}}
\newcommand{\VariableTok}[1]{\textcolor[rgb]{0.00,0.00,0.00}{#1}}
\newcommand{\VerbatimStringTok}[1]{\textcolor[rgb]{0.31,0.60,0.02}{#1}}
\newcommand{\WarningTok}[1]{\textcolor[rgb]{0.56,0.35,0.01}{\textbf{\textit{#1}}}}
\usepackage{graphicx}
\makeatletter
\newsavebox\pandoc@box
\newcommand*\pandocbounded[1]{% scales image to fit in text height/width
  \sbox\pandoc@box{#1}%
  \Gscale@div\@tempa{\textheight}{\dimexpr\ht\pandoc@box+\dp\pandoc@box\relax}%
  \Gscale@div\@tempb{\linewidth}{\wd\pandoc@box}%
  \ifdim\@tempb\p@<\@tempa\p@\let\@tempa\@tempb\fi% select the smaller of both
  \ifdim\@tempa\p@<\p@\scalebox{\@tempa}{\usebox\pandoc@box}%
  \else\usebox{\pandoc@box}%
  \fi%
}
% Set default figure placement to htbp
\def\fps@figure{htbp}
\makeatother
\setlength{\emergencystretch}{3em} % prevent overfull lines
\providecommand{\tightlist}{%
  \setlength{\itemsep}{0pt}\setlength{\parskip}{0pt}}
\usepackage{bookmark}
\IfFileExists{xurl.sty}{\usepackage{xurl}}{} % add URL line breaks if available
\urlstyle{same}
\hypersetup{
  pdftitle={R Notebook},
  hidelinks,
  pdfcreator={LaTeX via pandoc}}

\title{R Notebook}
\author{}
\date{\vspace{-2.5em}}

\begin{document}
\maketitle

\begin{Shaded}
\begin{Highlighting}[]
\CommentTok{\# install.packages("devtools")}
\NormalTok{devtools}\SpecialCharTok{::}\FunctionTok{install\_github}\NormalTok{(}\StringTok{"psrajoria/lexisbatch"}\NormalTok{)}
\end{Highlighting}
\end{Shaded}

\begin{verbatim}
## Skipping install of 'lexisbatch' from a github remote, the SHA1 (f2a508d7) has not changed since last install.
##   Use `force = TRUE` to force installation
\end{verbatim}

\begin{Shaded}
\begin{Highlighting}[]
\FunctionTok{library}\NormalTok{(lexisbatch)}
\end{Highlighting}
\end{Shaded}

🔑 Authentication

\begin{Shaded}
\begin{Highlighting}[]
\CommentTok{\# 1 Using environment variables (recommended)}

\CommentTok{\#Sys.setenv(LEXIS\_CLIENT\_ID = "your{-}client{-}id")}
\CommentTok{\#Sys.setenv(LEXIS\_CLIENT\_SECRET = "your{-}client{-}secret")}
\end{Highlighting}
\end{Shaded}

\begin{Shaded}
\begin{Highlighting}[]
\FunctionTok{lexis\_auth\_env}\NormalTok{()}
\end{Highlighting}
\end{Shaded}

\begin{Shaded}
\begin{Highlighting}[]
\CommentTok{\# 2 Inline (directly in code)}

\CommentTok{\# lexis\_auth("your{-}client{-}id", "your{-}client{-}secret")}
\end{Highlighting}
\end{Shaded}

📁 Output structure

All resumable functions store data under the working directory:
BatchNews// BatchCases// BatchDockets//

Each folder contains: checkpoint.rds page\_001.rds page\_002.rds
\ldots{} BatchCases\_id\_keyed.json

📰 BatchNews Example

\begin{Shaded}
\begin{Highlighting}[]
\FunctionTok{library}\NormalTok{(lexisbatch)}
\FunctionTok{lexis\_auth\_env}\NormalTok{()}

\CommentTok{\# Build query}
\NormalTok{q }\OtherTok{\textless{}{-}} \FunctionTok{bn\_new}\NormalTok{() }\SpecialCharTok{|\textgreater{}}
  \FunctionTok{bn\_search}\NormalTok{(}\StringTok{"title(theranos OR elizabeth holmes)"}\NormalTok{) }\SpecialCharTok{|\textgreater{}}
  \FunctionTok{bn\_filter\_date}\NormalTok{(}\StringTok{"2009{-}01{-}01"}\NormalTok{, }\StringTok{"2015{-}10{-}20"}\NormalTok{) }\SpecialCharTok{|\textgreater{}}
  \FunctionTok{bn\_expand}\NormalTok{(}\AttributeTok{document =} \ConstantTok{TRUE}\NormalTok{) }\SpecialCharTok{|\textgreater{}}
  \FunctionTok{bn\_top}\NormalTok{(}\DecValTok{50}\NormalTok{)}

\CommentTok{\# Debug final URL}
\FunctionTok{bn\_debug\_url}\NormalTok{(q)}
\end{Highlighting}
\end{Shaded}

\begin{verbatim}
## [1] "https://services-api.lexisnexis.com/v1/BatchNews?%24search=title%28theranos%20OR%20elizabeth%20holmes%29&%24select=ResultId%2CTitle%2CDate&%24filter=Date%20ge%202009-01-01%20and%20Date%20le%202015-10-20&%24expand=Document&%24top=50&%24count=true"
\end{verbatim}

\begin{Shaded}
\begin{Highlighting}[]
\CommentTok{\# Fetch automatically with checkpointing}
\NormalTok{news }\OtherTok{\textless{}{-}} \FunctionTok{bn\_fetch\_resumable\_auto}\NormalTok{(q)}
\end{Highlighting}
\end{Shaded}

\begin{verbatim}
## Resuming with 331 seen ids.
\end{verbatim}

\begin{verbatim}
##   | Page 1 (raw 50, new 0) | cum new=331/331
\end{verbatim}

\begin{verbatim}
##   | Page 2 (raw 50, new 0) | cum new=331/331
\end{verbatim}

\begin{verbatim}
##   | Page 3 (raw 50, new 0) | cum new=331/331
\end{verbatim}

\begin{verbatim}
##   | Page 4 (raw 50, new 0) | cum new=331/331
\end{verbatim}

\begin{verbatim}
##   | Page 5 (raw 50, new 0) | cum new=331/331
\end{verbatim}

\begin{verbatim}
##   | Page 6 (raw 50, new 0) | cum new=331/331
\end{verbatim}

\begin{verbatim}
##   | Page 7 (raw 31, new 0) | cum new=331/331
\end{verbatim}

\begin{Shaded}
\begin{Highlighting}[]
\CommentTok{\# Output stored in:}
\CommentTok{\# BatchNews/title\_theranos\_OR\_elizabeth\_holmes/}
\end{Highlighting}
\end{Shaded}

If the process stops or fails mid-run, just re-run:

\begin{Shaded}
\begin{Highlighting}[]
\CommentTok{\# news \textless{}{-} bn\_fetch\_resumable\_auto(q)}
\end{Highlighting}
\end{Shaded}

It will resume from the last saved page automatically.

⚖️ BatchCases Example

\begin{Shaded}
\begin{Highlighting}[]
\FunctionTok{library}\NormalTok{(lexisbatch)}
\FunctionTok{lexis\_auth\_env}\NormalTok{()}

\NormalTok{q }\OtherTok{\textless{}{-}} \FunctionTok{bc\_new}\NormalTok{() }\SpecialCharTok{|\textgreater{}}
  \FunctionTok{bc\_search}\NormalTok{(}\StringTok{"roe vs wade"}\NormalTok{) }\SpecialCharTok{|\textgreater{}}
  \FunctionTok{bc\_filter\_date}\NormalTok{(}\StringTok{"1965{-}01{-}01"}\NormalTok{,}\StringTok{"1980{-}12{-}31"}\NormalTok{) }\SpecialCharTok{|\textgreater{}}
  \FunctionTok{bc\_expand}\NormalTok{(}\AttributeTok{document =} \ConstantTok{TRUE}\NormalTok{) }\SpecialCharTok{|\textgreater{}}
  \FunctionTok{bc\_top}\NormalTok{(}\DecValTok{50}\NormalTok{)}

\FunctionTok{bc\_run\_resumable\_auto}\NormalTok{(q)}
\end{Highlighting}
\end{Shaded}

\begin{verbatim}
## Resuming BatchCases with 11 seen ids.
\end{verbatim}

\begin{verbatim}
##   | Page 1 (raw 11, new 0) | seen=11/11
\end{verbatim}

\begin{verbatim}
## == Using inline XML from column: Document.Content
\end{verbatim}

\begin{verbatim}
## == Parsing XML → JSON ...
\end{verbatim}

\begin{verbatim}
## # A tibble: 11 x 9
##    ResultId               Title Date  Document.DocumentId Document.DocumentIdT~1
##    <chr>                  <chr> <chr> <chr>               <chr>                 
##  1 urn:contentItem:3S4X-~ ""    1973~ /shared/document/c~ DocFullPath           
##  2 urn:contentItem:3S4X-~ ""    1973~ /shared/document/c~ DocFullPath           
##  3 urn:contentItem:3S4V-~ ""    1970~ /shared/document/c~ DocFullPath           
##  4 urn:contentItem:3S4X-~ ""    1971~ /shared/document/c~ DocFullPath           
##  5 urn:contentItem:3S4X-~ ""    1971~ /shared/document/c~ DocFullPath           
##  6 urn:contentItem:3S4X-~ ""    1972~ /shared/document/c~ DocFullPath           
##  7 urn:contentItem:3S4X-~ ""    1971~ /shared/document/c~ DocFullPath           
##  8 urn:contentItem:3S4X-~ ""    1971~ /shared/document/c~ DocFullPath           
##  9 urn:contentItem:3S4X-~ ""    1971~ /shared/document/c~ DocFullPath           
## 10 urn:contentItem:3S4X-~ ""    1972~ /shared/document/c~ DocFullPath           
## 11 urn:contentItem:3S3J-~ ""    1968~ /shared/document/c~ DocFullPath           
## # i abbreviated name: 1: Document.DocumentIdType
## # i 4 more variables: Document.Content <chr>, Document.Citation <chr>,
## #   Document.GenAIEnabled <lgl>, xml <chr>
\end{verbatim}

📜 BatchDockets Example

\begin{Shaded}
\begin{Highlighting}[]
\FunctionTok{library}\NormalTok{(lexisbatch)}
\FunctionTok{lexis\_auth\_env}\NormalTok{()}

\NormalTok{qd }\OtherTok{\textless{}{-}} \FunctionTok{bd\_new}\NormalTok{() }\SpecialCharTok{|\textgreater{}}
  \FunctionTok{bd\_search}\NormalTok{(}\StringTok{"1:22cv859"}\NormalTok{) }\SpecialCharTok{|\textgreater{}}
  \FunctionTok{bd\_filter\_date}\NormalTok{(}\StringTok{"2022{-}01{-}01"}\NormalTok{,}\StringTok{"2022{-}12{-}31"}\NormalTok{) }\SpecialCharTok{|\textgreater{}}
  \FunctionTok{bd\_top}\NormalTok{(}\DecValTok{50}\NormalTok{)}


\NormalTok{dck }\OtherTok{\textless{}{-}} \FunctionTok{bd\_run\_full\_resumable}\NormalTok{(qd)}
\end{Highlighting}
\end{Shaded}

\begin{verbatim}
## Resuming dockets with 21 seen IDs.
\end{verbatim}

\begin{verbatim}
## [DEBUG URL] https://services-api.lexisnexis.com/v1/BatchDockets?%24search=1%3A22cv859&%24select=ResultId&%24filter=Date%20ge%202022-01-01T00%3A00%3A00Z%20and%20Date%20le%202022-12-31T23%3A59%3A59Z&%24top=50&%24count=true
\end{verbatim}

\begin{verbatim}
##   | IDs page 1 | raw=21, new=0, seen now=21/21
\end{verbatim}

\begin{verbatim}
## No new IDs → no new XML to fetch.
\end{verbatim}

\begin{verbatim}
## == Parsing XML → JSON for all saved dockets ...
\end{verbatim}

\begin{center}\rule{0.5\linewidth}{0.5pt}\end{center}

🧠 Common Parameters Argument Description page\_sleep Pause between API
pages to respect rate limits retry, retry\_wait Retry on HTTP
429/500/503 fetch\_xml Whether to download XML documents parse\_xml
Whether to parse XML into JSON parallel\_xml Fetch XML concurrently
using furrr workers Number of parallel workers base\_dir Custom output
folder write\_json File path to save parsed JSON

💾 Reading saved data

You can reload data at any time:

\begin{Shaded}
\begin{Highlighting}[]
\NormalTok{rows }\OtherTok{\textless{}{-}} \FunctionTok{readRDS}\NormalTok{(}\StringTok{"BatchNews/title\_theranos\_OR\_elizabeth\_holmes\_/page\_001.rds"}\NormalTok{)}
\NormalTok{json }\OtherTok{\textless{}{-}}\NormalTok{ jsonlite}\SpecialCharTok{::}\FunctionTok{read\_json}\NormalTok{(}\StringTok{"BatchCases/roe\_vs\_wade/roe\_vs\_wade\_id\_keyed.json"}\NormalTok{)}
\end{Highlighting}
\end{Shaded}


\end{document}
